\documentclass[t,usepdftitle=false]{beamer}

\input{../../../tex-beamer-custom/preamble.tex}

\title[NLA for CS and IE -- Lecture 00]{Numerical Linear Algebra\\for Computational Science and Information Engineering}
\subtitle{\vspace{.3cm}Lecture 00\\Introduction}
\hypersetup{pdftitle={NLA-for-CS-and-IE\_Lecture00}}

\date[Summer 2025]{Summer 2025}

\author[nicolas.venkovic@tum.de]{Nicolas Venkovic\\{\small nicolas.venkovic@tum.de}}
\institute[]{Group of Computational Mathematics\\School of Computation, Information and Technology\\Technical University of Munich}

\titlegraphic{\vspace{0cm}\includegraphics[height=1.1cm]{../../../logos/TUM-logo.png}}

\begin{document}
	
\begin{frame}[noframenumbering, plain]
	\maketitle
\end{frame}
	
\myoutlineframe

\section{Numerical linear algebra:\\The "Why?", the "What?" and "What's next?"}

% Slide 01
\begin{frame}{Why do we need numerical linear algebra?}
\textbf{Prevalence of linear algebra in research and industry}:
\begin{itemize}
\item[-] Scientific computing (often explicitly related to problems of linear algebra)
\item[-] Simulations (e.g., discretized PDEs, ODEs, DAEs, ...)
\item[-] Machine learning (e.g., PCA, RKHS, ...)
\item[-] Optimization
\end{itemize}
\textbf{Methods taught in linear algebra}:
\begin{itemize}
\item[-] Solving linear systems: Cramer's rule and Gaussian elimination
\item[-] Matrix factorizations: Cholesky, LU and QR
\item[-] Eigenvalue problems: Root-finding of characteristic polynomials
\end{itemize}
\textbf{Limitations of methods taught in linear algebra}:
\begin{itemize}
\item[-] \textbf{{\color{red}{No general exact method for eigenvalues of matrices}}} larger than 4x4
\item[-] \textbf{{\color{red}{Computational infeasibility}}} of analytical methods \textbf{{\color{red}{for large systems}}}
\item[-] \textbf{{\color{red}{Numerical instability}}} leading to significant \textbf{{\color{red}{errors}}}
\item[-] \textbf{{\color{red}{Inability to handle special structures}}} (e.g., sparsity, implicitness)
\end{itemize}
\begin{center}
\fbox{\begin{minipage}{0.8\textwidth}\begin{center}
\textbf{Need for methods to solve challenging problems efficiently}
\end{center}\end{minipage}}
\end{center}
\end{frame}

% Slide 02
\begin{frame}{What is numerical linear algebra?}
\textbf{What does numerical numerical linear algebra (NLA) address}:
\begin{itemize}
\item[-] $\!$Efficient algorithms for large-scale problems of linear algebra
\item[-] $\!$Techniques for maintaining numerical stability and accuracy
\item[-] $\!$Methods for exploiting matrix structure (e.g., sparsity, symmetry, low-rank)
\item[-] $\!$Matrix-free formulations for implicit operators
\item[-] $\!$Error analysis and conditioning 
\end{itemize}
\textbf{Scope of NLA}:
\begin{itemize}
\item[-] $\!$Focuses on both theoretical analysis (e.g., convergence, conditioning and stability) and practical implementation.
\item[-] $\!$Incorporates aspects of performance and computer architecture for method\\
$\!$development, $\!$e.g.,$\hspace{-1cm}$
\begin{itemize}
\item[-] Repurposing solvers into preconditioners for higher performance.\vspace{.05cm}
\item[-] Designing sparse data structures and BLAS kernels to optimize memory usage.
\end{itemize}
\end{itemize}
\textbf{Key branches of NLA}:
\begin{itemize}
\item[-] $\!$\textbf{Direct methods} vs \textbf{iterative methods} (stationary vs Krylov)
\item[-] $\!$\textbf{Dense matrices} vs \textbf{sparse matrices} and \textbf{matrix-free operators}
\end{itemize}
\end{frame}

% Slide 03
\begin{frame}{What's coming next in numerical linear algebra?}
Recent developments in NLA research show promise for significant performance improvements, but face challenges in widespread adoption:
\begin{itemize}
\item \textbf{Randomization}: Randomly reduces problem dimensions while preserving inherent structure with high probability.
Associated challenges are
\begin{itemize}
\item[-] Ensuring convergence and assessing stability.\vspace{.08cm}
\item[-] Defining and implementing reliably amortized sketching procedures.
\end{itemize}
\item \textbf{Communication avoidance}: Redesigns algorithms to minimize data movement between processors or memory hierarchies.
Challenges are
\begin{itemize}
\item[-] Maintaining numerical stability with reduced communication.\vspace{.08cm}
%\item[-] Balancing communication reduction with increased computational complexity.
\end{itemize}
\item \textbf{Mixed precision}: Utilizes different numerical precisions within a single algorithm to optimize performance.
Associated challenges are
\begin{itemize}
\item[-] Ensuring overall stability and accuracy with lower precision components.\vspace{.08cm}
\item[-] Developing adaptive strategies for precision selection.
\end{itemize}
\item \textbf{Extension of state-of-the art methods to tensor data}: Assessing convergence and stability when using low-rank approximation of tensor data, e.g., Tucker and Tensor-Train formats.
\end{itemize}
\end{frame}

\section{Computational mathematics @ TUM}

% Slide 04
\begin{frame}{Computational mathematics @ TUM, Campus Heilbronn}
\begin{itemize}
\item $\!$Chair$\!$ of  COmputational$\!$  MAthematics$\!$  (COMA)$\!$ @$\!$ TUM: Prof.$\!$ \textbf{Hartwig Anzt}$\hspace{-1cm}$
\begin{itemize}
\item[-] Professor @ TUM since 01/2024\vspace{.04cm}
\item[-] Director of the Innovative Computing Lab @ University of Tennessee (2022-23)$\hspace{-1cm}$\vspace{.04cm}
\item[-] Junior Professor @ KIT (2021-22)\vspace{.04cm}
\end{itemize}
\item $\!$Recent and ongoing PhD theses directed by Hartwig:\vspace{.05cm}
\begin{itemize}
\item[-] Symbolic LU factorizations on GPUs (\textbf{Tobias Ribizel}, ongoing)\vspace{.05cm}
\item[-] Scalable domain decomposition on GPUs (\textbf{Fritz Göbel}, ongoing)\vspace{.05cm}
\item[-] Numerical $\!$compression $\!$in $\!$scientific $\!$computing $\!$(\textbf{Thomas Grützmacher}, $\!$ongoing)$\hspace{-1cm}$\vspace{.05cm}
\item[-] Mixed precision algebraic multigrids on GPUs (\textbf{Yu-Hsiang Mike Tsai}, 2024)\vspace{.05cm}
\item[-] Asynchronous and batched iterative solvers on GPUs (\textbf{Pratik Nayak}, 2023)
\end{itemize}
\item $\!$Ongoing projects in COMA:\vspace{.05cm}
\begin{itemize}
\item[-] \textbf{Ginkgo}: Portable high-performance numerical linear algebra library\vspace{.05cm}
\item[-] \textbf{Sparse BLAS}: Basic linear algebra subprograms for sparse matrices\vspace{.05cm}
\item[-] \textbf{ICON}: Large legacy codebase for climate modeling\vspace{.05cm}
\item[-] \textbf{OGL/OpenFOAM}: Large legacy codebase for computational fluid dynamics\vspace{.05cm}
\item[-] \textbf{MicroCARD}: Numerical modeling of cardiac electrophysiology\vspace{.05cm}
\item[-] \textbf{PDExa}: Optimized software methods and technologies for PDEs\vspace{.05cm}
\item[-] \textbf{nekRS}: Fast and scaleable computational fluid dynamics software package
\end{itemize}
\end{itemize}
\end{frame}

% Slide 05
\begin{frame}{Computational mathematics @ TUM, Campus Heilbronn, cont'd}
\begin{itemize}
\item Course instructor:\vspace{-.2cm}
\begin{center}\includegraphics[height=2cm]{images/2024-Venkovic-headshot.jpeg}\end{center}\vspace{.15cm}
\textbf{Nicolas Venkovic}, Postdoc. in COMA @ TUM%\vspace{.3cm}
\begin{itemize}
\item[-] Postdoctoral Researcher @ TUM since 07/2024\vspace{.07cm}
\item[-] Software Developer @ NXP Semiconductors (09/2022 - 04/2024)\vspace{.07cm}
\item[-] PhD in Applied Mathematics \& Scientific Computing @ Cerfacs (09/2023)\vspace{.07cm}
\item[-] MSE in Applied Mathematics \& Statistics @ Johns Hopkins University (2018)\vspace{.07cm}
\end{itemize}
\item Recent and ongoing research efforts:
\item[] Randomized Short-Recurrence Iterative Methods for the Approximation of:\\
\begin{itemize}\normalsize
\item[-] Matrix Inverses, Eigenpairs, Low-rank and Non-Negative Factorizations.\vspace{.07cm}
\end{itemize}
\item[] With applications to:
\begin{itemize}\normalsize
\item[-] Stochastic Preconditioning, Matrix Recovery, Parallelization, \dots
\end{itemize}
\end{itemize}
\end{frame}


\section{Class overview}

% Slide 06
\begin{frame}{Objectives}
Upon completion of the class, students should:
\begin{itemize}
\item Understand why numerical linear algebra (NLA) is essential in practice, and what it encompasses.
\item Be familiar with the challenges of the main problems and procedures of NLA, such as orthogonalization, least-squares solving, factorization, linear solving, preconditioning, eigenvalue solving, ...
\item Understand the definitions, implementations and main properties of the methods presented in class for solving these problems.
\item Recognize some pathological behaviors of these methods and be able to explain unwanted numerical behaviors.
\item Know the strengths and limits of the presented methods, and know when to use them in practice.
\item Have the foundational background to explore methods for solving NLA problems beyond those covered in class.
\end{itemize}
\end{frame}

% Slide 07
\begin{frame}{Content}
\begin{itemize}
\vspace{-.1cm}
\item We will cover the following core topics:%\vspace{.04cm}
\begin{itemize}
\item[-] \texttt{Lecture 01}: Essentials of linear algebra\vspace{.02cm}
\item[-] \texttt{Lecture 02}: Essentials of the Julia language\vspace{.02cm}
\item[-] \texttt{Lecture 03}: Floating-point arithmetic and error analysis\vspace{.02cm}
\item[-] \texttt{Lecture 04}: Direct methods for dense linear systems\vspace{.02cm}
\item[-] \texttt{Lecture 05}: Sparse data structures and basic linear algebra subprograms\vspace{.02cm}
\item[-] \texttt{Lecture 06}: Introduction to direct methods for sparse linear systems\vspace{.02cm}
\item[-] \texttt{Lecture 07}: Orthogonalization and least-squares problems\vspace{.02cm}
\item[-] \texttt{Lecture 08}: Basic iterative methods for linear systems\vspace{.02cm}
\item[-] \texttt{Lecture 09}: Basic iterative methods for eigenvalue problems\vspace{.02cm}
\item[-] \texttt{Lecture 10}: Locally optimal block preconditioned conjugate gradient\vspace{.02cm}
\item[-] \texttt{Lecture 11}: Arnoldi and Lanczos procedures\vspace{.02cm}
\item[-] \texttt{Lecture 12}: Jacobi-Davidson method\vspace{.02cm}
\item[-] \texttt{Lecture 13}: Krylov subspace methods for linear systems\vspace{.02cm}
\item[-] \texttt{Lecture 14}: Preconditioned iterative methods for linear systems\vspace{.02cm}
\item[-] \texttt{Lecture 15}: Restarted Krylov subspace methods\vspace{.02cm}
\item[-] \texttt{Lecture 16}: Elements of randomized numerical linear algebra\vspace{.02cm}
\item[-] \texttt{Lecture 17}: Introduction to communication-avoiding algorithms\vspace{.02cm}
\item[-] \texttt{Lecture 18}: Matrix function evaluation
\end{itemize}
\end{itemize}
%\vfill
%\tiny{The duration of a lecture is not tied to a single class period.
%Some lectures will span less than one class, while others will cover multiple meetings.
%Several lectures can be covered within one meeting.}
\end{frame}

% Slide 08
\begin{frame}{Content, cont'd}
\begin{itemize}
\item If time permits, additional lectures may be picked among the following based on students' interests:
\begin{itemize}
\item[-] \texttt{Lecture E1}: Multigrid methods and domain decomposition\vspace{.04cm}
\item[-] \texttt{Lecture E2}: Multilinear algebra and tensor decompositions\vspace{.04cm}
\item[-] \texttt{Lecture E3}: Introduction to mixed precision algorithms
\end{itemize}
%\end{itemize}
%$\vspace{-.3cm}$\\
%\tiny{The duration of a lecture is not tied to a single class period.
%Some lectures will span less than one class, while others will cover multiple meetings.
%Several lectures can be covered within one meeting.}\vspace{.05cm}
%\begin{itemize}
%\normalsize
\item The duration of a lecture is not tied to a single class period.
Some lectures will span less than one class, while others will cover multiple meetings.
Several lectures can be covered within one meeting.
\item Most lectures will begin by establishing formal foundations, and some will be followed by practice sessions using \textbf{Julia notebooks} or \textbf{exercise sheets} to test, visualize, and deepen your understanding of the methods and concepts introduced.
\end{itemize}
\end{frame}

% Slide 09
\begin{frame}{Reading material}
\begin{itemize}
\item \textbf{Main references}:
\begin{itemize}
\item[-] \textbf{Lecture slides} and \textbf{Julia notebooks} uploaded gradually on {\color{blue}{\underline{\href{https://www.moodle.tum.de/}{Moodle}}}}, throughout the semester.\\
Content also uploaded at {\color{blue}{\underline{\href{https://venkovic.github.io/NLA-for-CS-and-IE}{venkovic.github.io/NLA-for-CS-and-IE}}}}.
\item[-]\begin{minipage}{0.2\textwidth}\vspace{.15cm}
\includegraphics[height=1.8cm]{images/Darve2021.jpg}
\end{minipage}
\hspace{-.8cm}
\begin{minipage}{0.75\textwidth}
Darve, E., \& Wootters, M. (2021). Numerical linear algebra with Julia.
Society for Industrial and Applied Mathematics (SIAM).${}^\dagger$
\end{minipage}\vspace{.3cm}
\item[-] GitHub repository: 
\begin{center}\url{https://github.com/EricDarve/numerical_linear_algebra}\end{center}\vspace{.3cm}
\end{itemize}
\item \textbf{Most used supplemental reference}:
\begin{itemize}
\item[-]\begin{minipage}{0.2\textwidth}\vspace{.15cm}
\includegraphics[height=1.8cm]{images/Saad2003.jpg}
\end{minipage}
\hspace{-.8cm}
\begin{minipage}{0.75\textwidth}
Saad, Y. (2003). Iterative methods for sparse linear systems. 
Society for Industrial and Applied Mathematics (SIAM).${}^\dagger$
\end{minipage}\vspace{.3cm}
\end{itemize}
\end{itemize}
\vfill
\tiny{${}^\dagger$ SIAM members get 30\% off book prices. SIAM membership is free for students.}
\end{frame}

% Slide 10
\begin{frame}{Reading material, cont'd\textsubscript{1}}
\begin{itemize}
\item \textbf{Other useful supplemental references}:
\begin{itemize}
\item[-]\begin{minipage}{0.2\textwidth}\vspace{.3cm}
\includegraphics[height=1.8cm]{images/Demmel1997.jpg}
\end{minipage}
\hspace{-.8cm}
\begin{minipage}{0.75\textwidth}
Demmel, J. W. (1997). Applied numerical linear algebra. 
Society for Industrial and Applied Mathematics (SIAM).${}^\dagger$
\end{minipage}\vspace{.3cm}
\item[-]\begin{minipage}{0.2\textwidth}\vspace{.3cm}
\includegraphics[height=1.8cm]{images/Golub2013.jpg}
\end{minipage}
\hspace{-.8cm}
\begin{minipage}{0.75\textwidth}
Golub, G. H., \& Van Loan, C. F. (2013). Matrix computations. 
JHU press.
\end{minipage}\vspace{.3cm}
\item[-]\begin{minipage}{0.2\textwidth}\vspace{.3cm}
\includegraphics[height=1.8cm]{images/Greenbaum1997.jpg}
\end{minipage}
\hspace{-.8cm}
\begin{minipage}{0.75\textwidth}
Greenbaum, A. (1997). Iterative methods for solving linear systems. 
Society for Industrial and Applied Mathematics (SIAM).${}^\dagger$
\end{minipage}\vspace{.3cm}
\end{itemize}
\end{itemize}
\vfill
\tiny{${}^\dagger$ SIAM members get 30\% off book prices. SIAM membership is free for students.}
\end{frame}

% Slide 11
\begin{frame}{Reading material, cont'd\textsubscript{2}}
\begin{itemize}
\item[]
\begin{itemize}
\item[-]\begin{minipage}{0.2\textwidth}\vspace{.3cm}
\includegraphics[height=1.8cm]{images/Higham2002.jpg}
\end{minipage}
\hspace{-.8cm}
\begin{minipage}{0.75\textwidth}
Higham, N. J. (2002). Accuracy and stability of numerical algorithms. 
Society for Industrial and Applied Mathematics (SIAM).${}^\dagger$
\end{minipage}\vspace{.3cm}
\item[-]\begin{minipage}{0.2\textwidth}\vspace{.3cm}
\includegraphics[height=1.8cm]{images/Trefethen2022.jpg}
\end{minipage}
\hspace{-.8cm}
\begin{minipage}{0.75\textwidth}
Trefethen, L. N., \& Bau, D. (2022). Numerical linear algebra. 
Society for Industrial and Applied Mathematics (SIAM).${}^\dagger$
\end{minipage}
\end{itemize}
\end{itemize}
\vspace{3.3cm}
\tiny{${}^\dagger$ SIAM members get 30\% off book prices. SIAM membership is free for students.}
\end{frame}

% Slide 12
\begin{frame}{Course evaluation}
\begin{itemize}
\item \textbf{Final exam (determines base grade)}:\vspace{.2cm}
\begin{center}
\fbox{\begin{minipage}{0.65\textwidth}\begin{center}
\textbf{10:00$\,$-11:30}\\
\textbf{Thursday, 31 July 2025}\\
\textbf{Campus Heilbronn}\\
\textbf{D.2.01}
\end{center}\end{minipage}}
\end{center}$\vspace{0cm}$\\
\begin{itemize}\normalsize
\item[-] Questions in relation to the homework problems, material presented in class, and practice sessions.
\end{itemize}
\item \textbf{Homework policy}:
\begin{itemize}\normalsize
\item[-] Submit your solution to one problem per lecture.\vspace{.05cm}
\item[-] Deadline: 2 weeks after the end of the lecture.\vspace{.15cm}
\item[-] Regular submissions upgrade the final grade by one step.
\begin{itemize}\normalsize
\item[-] Example: 2.0 becomes 1.7.\vspace{.05cm}
\item[-] Notes: This upgrade applies to passing grades only.\\
\hspace{1.2cm}1.0 remains the highest possible grade.
\end{itemize}
\end{itemize}
\end{itemize}
\end{frame}

% Slide 12
\begin{frame}{Course evaluation}
\addtocounter{framenumber}{-1}
\begin{itemize}
\item \textbf{Retake exam}:\vspace{.2cm}
\begin{center}
\fbox{\begin{minipage}{0.65\textwidth}\begin{center}
\textbf{10:00$\,$-11:30}\\
\textbf{Monday, 29 September 2025}\\
\textbf{Campus Heilbronn}\\
\textbf{D.2.01}
\end{center}\end{minipage}}
\end{center}$\vspace{0cm}$\\
\begin{itemize}\normalsize
\item[-] Questions in relation to the homework problems, material presented in class, and practice sessions.
\end{itemize}
\item \textbf{Homework policy}:
\begin{itemize}\normalsize
\item[-] Submit your solution to one problem per lecture.\vspace{.05cm}
\item[-] Deadline: 2 weeks after the end of the lecture.\vspace{.15cm}
\item[-] Regular submissions upgrade the final grade by one step.
\begin{itemize}\normalsize
\item[-] Example: 2.0 becomes 1.7.\vspace{.05cm}
\item[-] Notes: This upgrade applies to passing grades only.\\
\hspace{1.2cm}1.0 remains the highest possible grade.
\end{itemize}
\end{itemize}
\end{itemize}
\end{frame}

\section{Homework assignment}

% Slide 13
\begin{frame}{Homework assignment}
Send an email to me with the subject line \texttt{NLA-YourLastName} addressing the following points:\vspace{.1cm}
\begin{enumerate}
\item Briefly describe your background, if any, in numerical linear algebra, e.g.,
    \begin{itemize}\normalsize
    \item[-] Courses taken, practical experience, self-study, ...
    \end{itemize}
\item Identify a specific area of interest in numerical linear algebra:
    \begin{itemize}\normalsize
    \item[-] A problem, method or concept you either work on, want to understand better, or are curious about.
    \end{itemize}
\item Regarding the course syllabus:
    \begin{itemize}\normalsize
    \item[-] List 2-3 topics you're most excited to learn about.
    \item[-] Mention any topics you feel you already have a strong grasp of, or simply would not mind skipping.
    \end{itemize}
\item Suggest elective topics, listed in the slides or not, that you'd like to see covered, in case time permits.
\end{enumerate}
%\vspace{0.05<cm}
\textbf{Note:} Your responses could help adapt certain aspects of the course to the class's needs and interests.
\end{frame}

\end{document}
