\documentclass[t,usepdftitle=false]{beamer}

\input{~/Dropbox/Git/tex-beamer-custom/preamble.tex}

\title[NLA for CS and IE -- Lecture 16]{Numerical Linear Algebra\\for Computational Science and Information Engineering}
\subtitle{\vspace{.3cm}Lecture 16\\Elements of Randomized Numerical Linear Algebra}
\hypersetup{pdftitle={NLA-for-CS-and-IE\_Lecture16}}

\date[Summer 2025]{Summer 2025}

\author[nicolas.venkovic@tum.de]{Nicolas Venkovic\\{\small nicolas.venkovic@tum.de}}
\institute[]{Group of Computational Mathematics\\School of Computation, Information and Technology\\Technical University of Munich}

\titlegraphic{\vspace{0cm}\includegraphics[height=1.1cm]{../../../logos/TUM-logo.png}}

\begin{document}
	
\begin{frame}[noframenumbering, plain]
	\maketitle
\end{frame}
	
\myoutlineframe
	
% Slide 01
\begin{frame}{Introduction}
\begin{itemize}
\item \textbf{Why} randomized numerical linear algebra (\textbf{RandNLA})?
\begin{itemize}\normalsize
\item[-] \textbf{Reducing algorithmic complexity} and \textbf{memory footprint}
\item[-] \textbf{Improving} numerical \textbf{stability}
\end{itemize}
\item \textbf{What} does RandNLA address?
\begin{itemize}\normalsize
\item[-] \textbf{Mapping} of vectors and/or matrices onto \textbf{low dimensional spaces} using \textbf{randomized embeddings}, i.e., in a way that \textbf{approximately preserves} the \textbf{geometry} of transformed object \textbf{with high probability}.
\end{itemize}
\item \textbf{What} is the \textbf{scope} of RandNLA?
\begin{itemize}\normalsize
\item[-] Theory: Characterization of random embeddings, stability analysis, ...
\item[-] Implementation: Fast sketching, ...
\end{itemize}
\item Seminal works:
\begin{itemize}\normalsize
\item[-] Least-squares problems
\item[-] Compressed sensing, i.e., sparse signal recovery
\item[-] Randomized trace estimation
\item[-] Randomized SVD and randomized low-rank
\item[-] Randomized Gram-Schmidt
\end{itemize}
\end{itemize}
\end{frame}	
	
\section{Random subspace embeddings}

% Slide 02
\begin{frame}{(Deterministic) subspace embeddings}
\begin{itemize}
\item Subspace embeddings can offer a means to reduce dimension of high-dimensional data.
\begin{definition}[Subspace embedding]
\begin{itemize}
\item[-] A \textbf{subspace embedding} with \textbf{distortion} $\varepsilon\in(0,1)$, or a $\varepsilon$-embedding, is a \textbf{linear map} $x\in\mathbb{F}^n\mapsto \Theta x\in\mathbb{F}^d$
which \textbf{embeds} $\mathbb{F}^d$ \textbf{into} $\mathcal{E}\subseteq\mathbb{F}^n$ , i.e.,
\begin{align*}
(1-\varepsilon)\|x\|\leq \|\Theta x\|\leq (1+\varepsilon)\|x\|
\;\forall\;x\in\mathcal{E}.
\end{align*}
\item[-] Such maps, also referred to as \textbf{quasi-isometries}, require $d\geq \mathrm{dim}(\mathcal{E})$.
\end{itemize}
\end{definition}
\item In practice, we want to achieve \textbf{significant dimension reduction}, i.e., $d\ll n$, but this \textbf{limits the possible dimension of} $\mathcal{E}$.
\end{itemize}
\end{frame}

% Slide 03
\begin{frame}{(Deterministic) subspace embeddings, cont'd}
\begin{definition}[Subspace embedding]
\begin{itemize}
\item[-] A \textbf{subspace embedding} with \textbf{distortion} $\varepsilon\in(0,1)$, or a $\varepsilon$-embedding, is a linear map $x\in\mathbb{F}^n\mapsto \Theta x\in\mathbb{F}^d$
which \textbf{embeds} $\mathbb{F}^d$ \textbf{into} $\mathcal{E}\subseteq\mathbb{F}^n$ , i.e.,
\begin{align*}
(1-\varepsilon)\|x\|\leq \|\Theta x\|\leq (1+\varepsilon)\|x\|
\;\forall\;x\in\mathcal{E}.
\end{align*}
\item[-] Such maps, also referred to as \textbf{quasi-isometries}, require $d\geq \mathrm{dim}(\mathcal{E})$.
\end{itemize}
\end{definition}
\begin{itemize}
\item Randomization allows to *.
\end{itemize}
\end{frame}

% Slide 03	
\begin{frame}{Randomized subspace embeddings}
The limitation of $d$ on the dimension of the embedding subspace $\mathcal{E}\subset \mathbb{F}^n$ is achieved by 
\begin{definition}[Randomized subspace embedding]
\begin{itemize}
\item[-] A \textbf{random subspace embedding} with \textbf{distortion} $\varepsilon\in(0,1)$ is a linear map $x\in\mathbb{F}^n\mapsto \Theta x\in\mathbb{F}^d$
which \textbf{embeds} $\mathbb{F}^d$ \textbf{into} $\mathcal{E}\subseteq\mathbb{F}^n$ , i.e.,
\begin{align*}
\mathrm{Pr}
\left\{
(1-\varepsilon)\|x\|\leq \|\Theta x\|\leq (1+\varepsilon)\|x\|
\right\}\geq 1-\delta
\end{align*}
\item[-] Such maps, also referred to as \textbf{quasi-isometries}, require $d\geq \mathrm{dim}(\mathcal{E})$.
\end{itemize}
\end{definition}
\begin{itemize}
\item stuff
\end{itemize}
\end{frame}


\section{Sketching methods}

% Slide *
\begin{frame}{Content}
\begin{itemize}
\item Counting sketch
\item SASO
\item Subsampled trigonometric transforms
\end{itemize}
\end{frame}

\section{Randomized low-rank}

% Slide *
\begin{frame}{Content}
\begin{itemize}
\item Content
\end{itemize}
\end{frame}




\end{document}
